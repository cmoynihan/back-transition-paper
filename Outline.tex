%        File: Outline.tex
%     Created: Wed Jun 27 07:00 AM 2018 P
% Last Change: Wed Jun 27 07:00 AM 2018 P
%
\documentclass[a4paper]{article}
\usepackage[margin = 1in]{geometry}
\begin{document}
\title{H-L Back Transition Paper Outline} %This will obviously need to be updated
\author{C. Moynihan, T.M. Wilks, D. Eldon, O. Meneghini, S. Smith, X.Q. Xu}
\date{}
\maketitle

\section{Introduction}

\begin{itemize}
	\item 
\end{itemize}

\section{Probably Something to do with back-transitions}

\section{Data to be collected (as of now)}

\begin{itemize}
	\item Look to see if the dominant mode shifts to higher mode number as rotation is decreased. With all other things remaining the same, this could be an indicator of softer transistions because higher mode numbers tend to penetrate less deeply into the plasma and will release less energy and reduce the pedestal height less. 
	\item Can change the radial electric field 
	\item Can change density in the pedestal 		
\end{itemize}

\section{Visuals}
\begin{itemize}
	\item Plot of growth rate as a function of mode number. Could show how mode number peak changes( in width, center, etc.) as a function of change in parameters (density, e-field, rotation) 
	\item Plot of some parameter as a function of $\psi_{N}$ (As seen in Eldon/Snyder papers for change in $T_{e}$ before and after an ELM)
	\item Will most likely want some polodial slice, or maybe a subplot ( 1x2 or 2x2) showing the differences between parameters (rotation, density, e-field)
		
\end{itemize}<++>

\end{document}


